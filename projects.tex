% das Papierformat zuerst
\documentclass[a4paper, 12pt]{article}

\usepackage[pdftex]{graphicx}

\usepackage{fancyvrb}

\usepackage[left=35mm,right=35mm,top=25mm,bottom=25mm]{geometry}

\usepackage[utf8]{inputenc}
\usepackage[T1]{fontenc}

\usepackage[english]{babel}

\usepackage[hidelinks]{hyperref}

% für bilder
\usepackage{graphicx}
% für floats (figures und so)
\usepackage{float}

% dfrac nutzen können
\usepackage{amsmath}
% um mathbb nutzen zu können
\usepackage{amssymb}

% Für Block-Kommentare
\usepackage{comment}

% Ob Kommentare angezeigt werden sollen oder nicht
%\includecomment{comment}
\excludecomment{comment}

% für Umrandungen
\usepackage{framed}

% Für Fußnoten in Tabellen
\usepackage{tablefootnote}

% Für vernünftige interne Verweise
\usepackage{titleref}

% Damit das Einbinden von UML-Diagrammen von Natur aus funktioniert.
\DeclareGraphicsRule{.1}{mps}{*}{}

% Setzt die Tiefe von Sections...
\setcounter{secnumdepth}{4}

% Setzt die Tiefe des Inhaltsverzeichnisses
\setcounter{tocdepth}{4}

% Damit Bilder und Text nebeneinander sein können
\usepackage[pdftex]{graphicx}
\usepackage{sidecap}


\hyphenation{trans-parancy}
\hyphenation{Autodesk}
\hyphenation{descent}
\hyphenation{computer-graphics}
\hyphenation{Autodesk}
\hyphenation{Softimage}
\hyphenation{different}
\hyphenation{DevFest}

% hier beginnt das Dokument
\begin{document}
	
\thispagestyle{empty}

% =======================================================================
% Titelseite


% nicht zu parsende Zeichen
\begin{verbatim}





\end{verbatim}

\begin{center}
\textbf{\LARGE{Software Projects}}
\end{center}

% nicht zu parsende Zeichen
\begin{verbatim}


\end{verbatim}

\begin{center}
\textbf{\large{List and description of software projects I created in context of my studies and in my private time}}
\end{center}

% nicht zu parsende Zeichen
\begin{verbatim}












\end{verbatim}

\begin{flushleft}
\begin{tabular}{llll}

\textbf{Subject:}
& & & Software Projects
\\
\\
\textbf{Author:}
& & & Maurice Tollmien -- maurice.tollmien@gmail.com
\\
\\
\textbf{Last modification:}
& & & \today



\end{tabular}
\end{flushleft}


\newpage

% =======================================================================
% Inhaltsverzeichnis

\tableofcontents
\newpage

% -----------------------------------------------------------------------
% Introduction

\section{Introduction}

In context of job applications and interviews, it is not always clear, what skills and capabilities the candidate (in this case - me)
has and where the main technical competence lies. 
\newline
During the last years, it became apparent, that a simple list of relevant projects with short desciptions often circumvents and avoids such uncertainties.
\newline
\newline
With those projects, often linked with the correspondent repositories, I hope to create an impression of my strengths and 
capabilities.
\newline
\newline
Most of my more meaningful projects are uploaded or developed using the following GitHub repository:

\begin{center}
	\url{https://github.com/MauriceGit/}
\end{center}$\\$
All projects I designed, planned and implemented myself are explicitly published as Open Source for 
unrestricted further use.
\newline
\newline
The list of exercises and projects is not complete and rather serves to create a valid impression of the technical side of myself.
The majority of the listed projects were created in context of my studies (Computer Science - Bachelor and Master).
Several others I build in my private time.

\newpage

\section{Conference Talks}

I gave a Technical Talk on the Google-Developer-Group Conference (GDG DevFest, October 2016) in Hamburg, Germany on my current research and Master Thesis subject:  \textit{\textbf{Illumination of density volumes with Neural Networks}}.

\section{Computergraphics and Virtual Reality}

\subsection{Basic Computergraphics Projects}

2/3D programs in C, using the OpenGL interface and pipeline. Specific projects vary from different Lighting Computations
in 3D, Shadow Volumes and Transparancy to higher order Spline Interpolation.
\\
Later projects include physically correct approximated calculations for sphere movement on spline surfaces, collision 
detection and particle simulations to give some examples.
\\
Most later implementations make extended use of the OpenGL shader language GLSL for Vertex, Fragment and
Compute Shaders.

\subsection{Procedural Modelling}

Creation of a 3D model (from scratch) of a random item with Autodesk Softimage.
The resulting object file could then be used by different projects.
\\
\\
This project (including screenshots) is uploaded on my GitHub account:

\begin{center}
	\url{https://github.com/MauriceGit/Procedural_Modeling_Spyder}
\end{center}

\subsection{Cloth Simulation}

This project was created entirely in my private time. It implements basic cloth simulation using a particle 
based approach with virtual springs for structure. The cloth interacts with different objects like spheres,
quads, planes, as well as virtual wind. 
\\
This project (including screenshots and further information) is uploaded on my GitHub account:

\begin{center}
	\url{https://github.com/MauriceGit/Cloth_Simulation}
\end{center}

\subsection{Water Simulation with  Reflections/Refractions}\label{subsec:water}

Implementation of a pressure based water simulation, internally represented as a height field.
The surroundings are procedurally generated using a self implemented Perlin Noise.
\newpage $ $\newline
The reflections and refractions are entirely computed in screen space, using the OpenGL fragment
shader.
\\
\\
This project (including screenshots and further information) is uploaded on my GitHub account:

\begin{center}
	\url{https://github.com/MauriceGit/Water_Simulation}
\end{center}

\subsection{Particle Simulation on GPU}

Implementation of a particle simulation (Euler Integration) by using the OpenGL compute shader.
All computations, regarding the particles, are done by the GPU entirely. Leading to impressive 
speed and particle count.
\\
\\
Further information and source code is uploaded on my GitHub account:

\begin{center}
	\url{https://github.com/MauriceGit/Partikel_accelleration_on_GPU}
\end{center}

\subsection{Screen Space Ambient Occlusion (SSAO)}

A private project, implementing a version of Screen Space Ambient Occlusion (SSAO). Adjustments to the Ambient 
Occlusion are possible while running the sample. The implementation of SSAO is done using the
OpenGL fragment shader.
\\
\\
Further information and source code is uploaded on my GitHub account:

\begin{center}
	\url{https://github.com/MauriceGit/Screen_Space_Ambient_Occlusion}
\end{center}

\subsection{Low-Level Quaternion Library}

I designed a low level C library for common operations on Quaternions. The library is made completely 
Open Source and stand-alone.
\\
Next to the library, I included a small proof-of-concept program, that rotates an object using Quaternions.
\\
\\
The library can be accessed on my GitHub account:

\begin{center}
	\url{https://github.com/MauriceGit/Quaternion_Library}
\end{center}

\subsection{Virtual Reality Project}

The main goal of this project was to implement software to make a Head Mounted Display (HMD) compatible to Linux.
This included reverse engineering of the USB bitstream and interpret it as Quaternions for further use in arbitrary 
software. The final state of this project was a C library which could be used in any C program using the OpenGL
pipeline to create a real 3D experience using the HMD.

\subsection{Environment Mapping}

As extension to the implemented water simulation project (see \ref{subsec:water} \glqq\nameref{subsec:water}\grqq), I
presented a seminar on how environment mapping could be implemented and realized in virtual environments.
Mostly featuring screen space reflections and parallax corrected cubemaps.

\subsection{Delaunay and Voronoi Partition of Images}

As a private project, I implemented a software, which takes an image, extracts points according to the local
image contrast and triangulates them with a Delaunay triangulation (implemented completely from scratch).
The triangulation is then converted to Voronoi regions. The output are images with the Delaunay/Voronoi
regions colored in the mean color of the particular region.
\\
\\
Further information and source code is uploaded on my GitHub account:

\begin{center}
	\url{https://github.com/MauriceGit/Delaunay_Triangulation}
\end{center}

\section{Artificial Intelligence}

\subsection{Game AI (Bachelor Thesis)}\label{subsec:thesis}

In the course of my Bachelor thesis, I created a dissertation in a subdiscipline of artificial intelligence (AI) 
with a complex board game (Arimaa\footnote{\url{http://arimaa.com/arimaa/}}) as subject.
Next to the theoretical approach, I implemented most of the relevant algorithms such as Alpha-Beta Pruning, 
Transposition Tables, Zobrist Hashing, Move Ordering and Parallelization.
\\
The dissertation won an award for innovation (Wedeler Hochschulpreis).
\\
\\
The dissertation is published on the official Arimaa homepage: 

\begin{center}
	\url{http://arimaa.com/arimaa/papers/MauriceTollmien/Thesis.pdf}
\end{center}$\\$
The source code of the AI is uploaded and available on my GitHub account:

\begin{center}
	\url{https://github.com/MauriceGit/Bachelor-Thesis-Arimaa}
\end{center}

\subsection{Pure Logical Agent}

Design and implementation of a purely logical agent for the game of Wumpus World\footnote{\url{https://en.wikipedia.org/wiki/Hunt_the_Wumpus}}, which dynamically acts according
to the game situation. The resulting logical expression is then solved, using a SAT solver 
(Picosat\footnote{\url{https://hackage.haskell.org/package/picosat}}).
\\
The implementation is functional and written in Haskell. The project is uploaded on my GitHub account:

\begin{center}
	\url{https://github.com/MauriceGit/AI-Logical-Agent}
\end{center}

\subsection{Artificial Neural Network}

This project was created for university and proceeded in my private time. It contains a from-scratch implementation
of a backpropagation neural network for letter recognition. Several extensions like 
gradient descent and dynamic hidden layer count and size are implemented.
\\
In the end it turns out to be roughly on the same level of effectiveness and efficiency as common backpropagation neural networks
like PyBrain\footnote{\url{http://pybrain.org/}}.
\\
\\
The implementation as well as a presentation are uploaded on my GitHub account:

\begin{center}
	\url{https://github.com/MauriceGit/Artificial_Neural_Network_Character_Classification}
\end{center}

\section{IT Security}

\subsection{Seminar Paper on Compiler Backdoors}

Preparing and presenting a seminar paper on backdoors in software implementation, especially in 
a compiler itself.
\\
The seminar paper is written in english and published on the website of the university:

\begin{center}
	\url{http://www.fh-wedel.de/~gb/seminare/ws2012/tollmien_backdoors.pdf}
\end{center}

\subsection{Cryptography Project}

Extended workshop on cryptographic methods and algorithms. The workshop covered the practical approach as well as
the mathematical background of common cryptographic encryption, hashing and signature algorithms.
\\
Included subjects in this workshop were for example: DES, GOST, AES, message integrity/authenticity, random number
generators, number theory, asynchrone techniques like Diffie-Hellman, RSA, elliptic curve cryptography and one way
accumulators.
\\
\\
Every task and correspondent solution is uploaded on my GitHub account:

\begin{center}
	\url{https://github.com/MauriceGit/Cryptography_Workshop}
\end{center}

\newpage

\subsection{Secure File Encryption}

As a private project I implemented an interactive secure file encryption in Python. It alows encryption/decryption, read mode and write mode (using Linux \glq less\grq\  and \glq nano\grq\ tools). The encryption consists of hashing the password and AES.
\\
\\
The complete implementation can be viewed under the following link:

\begin{center}
	\url{https://github.com/MauriceGit/Secure_File_Encryption}
\end{center}


\subsection{Mental Poker}

In this project a group of students (myself included) extended the official Bouncy Castle\footnote{\url{https://www.bouncycastle.org/}} Crypto library with the SRA Algorithm which can be used
for mental poker. The main difference to the common RSA algorithm is its commutative behaviour regarding 
the encryption and decryption.
\\
\\
The SRA implementation is published under the following link:

\begin{center}
	\url{https://github.com/timpauls/bc-java}
\end{center}


\section{Programming Contests}

\subsection{Shortest path - 2013}

I took part in a programming contest with the goal, to find the shortest path through a set of points.
I used a mesh triangulation (Delaunay) from Computergraphics and was able to reach the 3rd rank
overall.

\subsection{2048 AI - 2014}

I participated in a programming contest with the main goal to implement a fast and efficient AI for the game
2048\footnote{\url{https://gabrielecirulli.github.io/2048/}}. I reused most of my AI algorithms from my bachelor 
thesis (see \ref{subsec:thesis} \glqq\nameref{subsec:thesis}\grqq) and was able to reach the 2nd rank overall.
\\
\\
The complete source code is uploaded on my GitHub account:

\begin{center}
	\url{https://github.com/MauriceGit/2048_Game_AI}
\end{center}

\newpage

\subsection{SameGame AI - 2015}

I participated in the programming contest for implementing an efficient AI for the game
SameGame\footnote{\url{https://de.wikipedia.org/wiki/SameGame}}. My implementation used a Monte-Carlo tree 
search. I was able to reach the 3rd rank overall.
\\
\\
The complete source code is uploaded on my GitHub account:

\begin{center}
	\url{https://github.com/MauriceGit/Programming_Contest_2015}
\end{center}


\subsection{2D Bin Packing}

For this year's programming contest, I implemented a paper\footnote{\glqq A two-level search algorithm for 2D rectangular	packing problem\grqq \ by Mao Chen and Wenqi Huang} on optimal 2D bin packing.
\newline
I participated in the contest with a multithreaded version of the implemented paper, parallel with a few simpler but faster algorithms. I was able to reach the 2nd rank overall.
\\
\\
The complete source code is uploaded on my GitHub account:

\begin{center}
	\url{https://github.com/MauriceGit/2D_Bin_Packing}
\end{center}

\subsection{Organisation of the Programming Contest 2016}\label{agar}

The programming contest for the next semester is organised by a fellow student and myself. 
The contest is a about a game similar to Agar.io\footnote{\url{http://agar.io/}}. All bots will play in real time against each other and will be challenged with different opponents, fields and situations.
\newline
All testing is done on a central server. Testers use our Middleware which handles all connections and data transfer to and from the server, making a participation very easy.
\newline
The programming contest has already started and is expected to run until the 9th November. At this time there will be an event and the final contest.
\newline
All information, including a live view of playing bots can be found under the following link. All information is in german as a courtesy for the participants.

\begin{center}
	\url{http://cagine.fh-wedel.de:8080/}
\end{center} $ $
\newline
The complete source code can be found here:

\begin{center}
	\url{https://github.com/hpatjens/Programmierwettbewerb}
\end{center}


\section{Other Projects}

\subsection{XBox controller interface in C}

I created a joystick interface for generic USB joysticks in C.
Using this library makes it trivial to include and use a USB XBox controller in a C program. The interface methods are adapted to fit perfectly in an OpenGL environment for object or camera control.
\\
\\
The complete source code is uploaded on my GitHub account:

\begin{center}
	\url{https://github.com/MauriceGit/XBox_Controller_Linux_Interface}
\end{center}

\subsection{Android App}

In cooperation with the wildlife park Eekholt\footnote{\url{http://www.wildpark-eekholt.de/}} we developed an Android 
App as a navigation help for visitors. 

\subsection{Lua Compiler}

As part of a team of several students, we developed a Lua Interpreter from scratch in Java. My part was mostly the 
code generation (Generate assembly like code from an abstract syntax tree) but I was involved in decision making
of the runtime environment and parser/scanner development as well. 

\subsection{Distributed Systems in Erlang}
 
Partly for university, partly as a private project, I developed a few algorithms for distributed systems (like 
bully algorithm, time control, mail delivery, ...) in Erlang.
\\
\\
The implementation of the bully algorihm is uploaded on my GitHub account:
\begin{center}
	\url{https://github.com/MauriceGit/Distributed_Systems_Bully_Algorithm}
\end{center}

 






\end{document}

